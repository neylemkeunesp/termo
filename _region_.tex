\message{ !name(v2.tex)}
\message{ !name(v2.tex) !offset(-2) }
\thispagestyle{headandfoot}
\begin{center} {\large Verifica��o II}
\end{center}
\vspace{0.5cm} Nome:\rule{14cm}{0.01cm} \\



\vspace{1 cm}
 
{\bf S\'o ser\~ao aceitas respostas devidamente justificadas.}

\vspace{1 cm}
\begin{questions}
  \question[2.0] Reduza as derivadas a express�es que contenham
  $\alpha$, $\kappa_T$ e $C_p$.
  \begin{parts}
  \item $$\left( \frac{\nabla T}{\nabla p} \right)_p$$
    \item $$\left( \frac{\nabla T}{\nabla V} \right)_U$$
      \end{parts}

\question[1.5] Considere as equa��es de estado:
$$
\frac{1}{T}=\frac{a}{u}+b v
$$
$$
\frac{p}{T}=\frac{c}{v}+f(u)
$$
\begin{parts}
\item Determine $f(u)$ e $u=u(s,v)$.
\item Determine a energia Livre de Helmholtz. 
\end{parts}



Selecione apenas tr�s das quest�es abaixo para fazer. Indique
explictamente quais voc� escolheu.

\question[1.5] Determine a envolt�ria complexa da fun��o:

$$
f(x)=x^4- a x^2 +b x
$$
onde $a$ e $b$ s�o constantes positivas.

\question[1.5] A baixas temperaturas, a energia livre de Helmholtz
molar $f(T,v)$ para o fluido de van der Waals pode ser aproximada por
duas express�es que descrevem a fase l�quida e a fase gasosa.
Para o l�quido vale:
$$f_L=-R T \ln (v-b)-\frac{a}{b}=\frac{a}{b^2}(v-b)+K$$
para a fase gasosa vale:
$$f_G=-R T \ln v +K$$
Fa�a uma transforma��o de Legendre e para obter a energia
livre de Gibbs. Determine a linha de coexist�ncia igualando as
energias livres do g�s e do l�quido.

\question[1.5] Considere a equa��o de Dieterici:
$$ p(v-b)=RT exp(-a/RTv)$$.
\begin{parts}
\item Determine em fun��o de $a$, $b$ e $c$ os valores da
  press�o, volume molar e temperatura cr�tica.
\item Escreva a equa��o de Dietererici em termos da press�o,
  volume molar e temperatura cr�tica.
\item Nas proximidades de $T_c$ temos que
$$p=p_o+A(v-v_o)$$
Determine o valor de $A$, $v_o$ obedece $p^{\prime\prime}(v_o)=0$ e
$p_o=p(v_o)$.
\end{parts}


\question[1.5] Considere o Hamiltoniano dado por:
$$H=J\sum_{i}^N \sigma_j^2$$ 
onde $\sigma_j=-1,0,1$.
\begin{parts}
\item Utilizando o ensemble microcan�nico determine a entropia $S$
  e o calor espec�fico a volume constante.
\item Repita os mesmos c�lculos utilizando o ensemble can�nico.
\end{parts}

\question[1.5] Considere o Hamiltoniano dado por:
$$H=\sum_{i}^N \frac{p_i^2}{2 m}+\frac{1}{2}m \omega^2 |x_i|$$ 
que representa part�culas em uma dimens�o.
\begin{parts}
\item Utilizando o ensemble can�nico determine a energia livre de
  Helmholtz e o calor espec�fico a volume constante.
\end{parts}

\question[1.5] Determine o coeficiente de virial $A(T)$ para um g�s 
cl�ssico de part�culas interagindo atrav�s do potencial:

 $$
V(r)= \left\{ \begin{array}{ll}
               V_M  & \mbox{se $0<r<r_o$} \\
               -\epsilon     & \mbox{ se $r_o<r<r_1$ }\\
                0 & \mbox{ se $r>r_1$ }
               \end{array}
     \right.
$$

\end{questions}


%%% Local Variables: 
%%% mode: latex
%%% TeX-master: "exame"
%%% End: 

\message{ !name(v2.tex) !offset(-109) }
