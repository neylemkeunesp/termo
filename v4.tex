
\thispagestyle{headandfoot}
\begin{center} {\large RER}
\end{center}
\vspace{0.5cm} Nome:\rule{14cm}{0.01cm} \\



\vspace{1 cm}
 
{\bf S\'o ser\~ao aceitas respostas devidamente justificadas.}

\vspace{1 cm}
\begin{questions}


\question[3.0] O n�cleo de um �tomo de hidrog�nio um proton possui um momento 
magn�tico.
Em um campo magn�tico, o proton possui dois estados de diferentes energias., 
spin para cima e  spin para baixo. Esta � a base da resson�ncia magn�tica 
nuclear para protons. As popula��es
 relativas s�o dadas pela distribui��o de Boltzmann. A diferen�a de 
energia entre os dois
estados � $\Delta\epsilon =g\mu_B$ onde $g=2.79$ e
 $\mu=5.05\times 10^{-24}$ J Tesla$^{-1}$. Para
um instrumento de resson�ncia nuclear, $B$=7 Tesla. Considere $T=300$ K.

\begin{parts}
  \item Calcule a fun��o parti��o.
  \item Calcule a diferen�a populacional:
$$\frac{|N_+-N_-|}{N_++N_-}$$
\end{parts}



\question[3.0] Um g�s de f�tons obedece as seguintes equa\c{c}�es:
$$U=bVT^4$$
 e
$$P=\frac{U}{3V}$$
\begin{parts}

\item Determine $S=S(U,V)$.
\item Determine $F=F(T,V)$.
\item Determine $C_V$ e $C_P$. 
\end{parts}



\question[2.0] Considere um ba� contendo 5 bolas duas
vervelhas e duas amarelas. Qual � a probabilidade
de se retirar duas bolas amarelas em sequ�ncia com 
reposi��o e sem reposi��o. 


\question[2.0] Uma parti��o separa volumes iguais
contendo $N$ part�culas do mesmo tipo na mesma temperatura. 
Usando um modelo de rede simples, calcule a multiplicidade 
do sistema com e sem uma barreira separando os dois sub-sistemas
e mostre que em ambos os casos os valores s�o equivalentes. 
Use a aproxima��o de Stirling. 

\end{questions}


%%% Local Variables: 
%%% mode: latex
%%% TeX-master: "exame"
%%% End: 
