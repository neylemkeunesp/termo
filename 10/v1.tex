\thispagestyle{headandfoot}
\begin{center} {\large Verifica��o I}
\end{center}
\vspace{0.5cm} Nome:\rule{14cm}{0.01cm} \\



\vspace{1 cm}
 
{\bf S\'o ser\~ao aceitas respostas devidamente justificadas.}

\vspace{1 cm}
\begin{questions}

\question[2.0] Um time de futebol ao empatar marca 
1 ponto, ao ganhar marca 3 e ao perder n�o marca nada. 
O Internacional, melhor time de todos os tempos, marcou no brasileir�o
de 2010 em m�dia 1.45 pontos por rodada. Assuma que o valor 
esperado da pontua��o por rodada seja 1.45. Usando o prin�ipio de m�xima 
entropia, determine a probabilidade de vit�ria, empate e derrota. 
Compare com os resultados experimentais: ($p_V=0.42$ $p_D=0.32$ $p_E=0.26$).

\question[2.0] Um sujeito joga todas as semanas na Megasena (ele deve 
acertar 6 dezenas entre os n�meros 1 e 50).
\begin{itemize}
\item Qual � a probabilidade dele acertar jogando apenas uma dezena?
\item Qual � a probabilidade dele ganhar ao menos uma vez 
ao longo de toda sua vida, jogando apenas uma dezena por semana 
toda a sua vida? Assuma que ele vai viver 80 anos.
\item Qual � a probabilidade dele ganhar uma �nica vez 
ao longo de toda sua vida, jogando apenas uma dezena por semana,
toda a sua vida? Assuma que ele vai viver 80 anos.

\end{itemize}

\question[2.5] Uma parti��o separa volumes iguais
contendo $N$ part�culas do mesmo tipo na mesma temperatura. 
Usando um modelo de rede simples, calcule a multiplicidade 
do sistema com e sem uma barreira separando os dois sub-sistemas
e mostre que em ambos os casos os valores s�o equivalentes. 
Use a aproxima��o de Stirling. 


\question[2.5] Um mole de um g�s de van der Waals � comprimido
de forma quase-est�tica do volume $V_1$ at� o volume $V_2$. 
Para um g�s deste tipo temos que:

$$p=\frac{RT}{V-b}-\frac{a}{V^2}$$

\begin{itemize}
\item Calcule o trabalho,
\item Compare com o trabalho realizado por um g�s ideal nas mesmas condi��es.
No limite de altas densidades. 
\end{itemize}
\end{questions}


%%% Local Variables: 
%%% mode: latex
%%% TeX-master: "exame"
%%% End: 
