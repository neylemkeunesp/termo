\thispagestyle{headandfoot}
\begin{center} {\large Verifica��o I}
\end{center}
\vspace{0.5cm} Nome:\rule{14cm}{0.01cm} \\



\vspace{1 cm}
 
{\bf S\'o ser\~ao aceitas respostas devidamente justificadas.}

\vspace{1 cm}
\begin{questions}

\question[1.5] Considere um ba\'u contendo 7 bolas duas
vervelhas e cinco amarelas. Qual \'e a probabilidade
de se retirar duas bolas amarelas em sequ\^encia com 
reposi\c{c}\~ao e sem reposi\c{c}\~ao. 

\question[1.5] Considere uma fun\c{c}\~ao de probabilidade 
localizada em [0,1] e dada por $q(x)=A(1-x^2)$. Determine:
\begin{itemize}
\item $A$ para que $q(x)$ seja normalizada.
\item O valor m\'edio de $x$ e $x^2$. 
\end{itemize}


\question[1.5] Lance uma moeda $8 N$ vezes. Qual \'e o  n\'umero
mais prov\'avel de caras � . Determine esta 
probabilidade e a probabilidade de obtermos apenas $N$
caras. Mostre usando a f\'ormula de Stirling que
a raz\~ao $p(N)/ p(4 N)$  tende a zero. 

 
\question[1.5] Considere um sistema com $N$ estados poss\'i{\i}veis.
Calcule a entropia quando:
\begin{itemize}
\item Um estado possui $p_i=1$.
\item Todos os estados s\~ao equiprov\'aveis.
\item Um quarto dos estados \'e  equiprov\'avel e a outra 
metade ocorre com $p_i=0$. Assuma $N$ divis\'{\i}vel por 4. 
\end{itemize}
 
\question[2.0] Uma parti\c{c}\~ao separa volumes iguais
contendo $N$ part\'{\i}culas do mesmo tipo na mesma temperatura. 
Usando um modelo de rede simples, calcule a multiplicidade 
do sistema com e sem uma barreira separando os dois sub-sistemas
e mostre que em ambos os casos os valores s\~ao equivalentes. 
Use a aproxima\c{c}\~ao de Stirling. 


\question[1.5] Um mole de um g\'as de van der Waals \'e comprimido
de forma quase-est\'atica do volume $V_1$ at\'e o volume $V_2$. 
Para um g\'as deste tipo temos que:

$$p=\frac{RT}{V-b}-\frac{a}{V^2}$$

\end{questions}


%%% Local Variables: 
%%% mode: latex
%%% TeX-master: "exame"
%%% End: 
