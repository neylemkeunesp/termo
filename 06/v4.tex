\thispagestyle{headandfoot}
\begin{center} {\large RER Termodin�mica}
\end{center}
\vspace{0.5cm} Nome:\rule{14cm}{0.01cm} \\



{\bf S\'o ser\~ao aceitas respostas devidamente justificadas.}

\vspace{1 cm}

\begin{questions}

\question[2.0] Obtenha a rela��o fundamental de um g�s ideal na representa��o 
de entalpia a partir da rela��o:
\[
S=N\left[s_o+c\ln \frac{U}{Nu_o}+R\ln \frac{V}{Nv_o}\right]
\]

\question[2.0] Um g�s ideal sofre uma expans�o adiab�tica a
  partir de um ponto $(V_1,p_1)$ at� atingir o volume $V_2$.

  \begin{parts}

  \item Determine o trabalho realizado.
  \item Determine o calor absorvido.
  \item Determine a varia��o de entropia.
  \end{parts}




  \question[2.0] Demonstre a rela��o de Slater
  \[
  \gamma_G=-\frac{1}{6}-\frac{v}{2 B_S}\frac{\partial B_S}{\partial v}
  \]
  a partir da rela��o entre $v_{som}$ e o m�dulo da
  elasticidade volum�trica $B_S$.  


\question[2.0] Dois corpos identicos possuem temperatura $T_1$ e
$T_2$.  Eles s�o colocados em contato t�rmico e atingem o estado
de equil�brio.  Ache o trabalho m�ximo que pode ser extra�do
destes corpos e a temperatura de equil�brio. Suponha $C_V$ constante. 



\question[2.0] Considere o Hamiltoniano dado por:
$$H=J\sum_{i}^N \sigma_j^2$$ 
onde $\sigma_j=-1,0,1$.
\begin{parts}
\item Utilizando o ensemble microcan�nico determine a entropia $S$
  e o calor espec�fico a volume constante.
\item Repita os mesmos c�lculos utilizando o ensemble can�nico.
\end{parts}


\end{questions}


%%% Local Variables: 
%%% mode: latex
%%% TeX-master: "exame"
%%% End: 
