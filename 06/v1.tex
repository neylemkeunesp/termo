\thispagestyle{headandfoot}
\begin{center} {\large Verifica��o I}
\end{center}
\vspace{0.5cm} Nome:\rule{14cm}{0.01cm} \\



\vspace{1 cm}
 
{\bf S\'o ser\~ao aceitas respostas devidamente justificadas.}

\vspace{1 cm}
\begin{questions}
  
  \question[2.0] Um g�s sofre um processo quase-est�tico e se
  expande a partir de um estado A caracterizado por um volume $V_o$ e
  uma press�o $p_o$ at� um volume $V_1$. Nesta expans�o a
  press�o varia com o volume de acordo com $p=p_oV_o^{5/3}V^{-5/3}$.


  \begin{parts}

  \item Determine a press�o $p_1$ correspondente ao estado B.
  \item O g�s teve sua energia aumentada ou diminu�da?
  \end{parts}

  \question[3.0] Um g�s ideal sofre uma expans�o adiab�tica a
  partir de um ponto $(V_1,p_1)$ at� atingir o volume $V_2$.

  \begin{parts}

  \item Determine o trabalho realizado.
  \item Determine o calor absorvido.
  \item Determine a varia��o de entropia.
  \end{parts}



\question[1.0] Determine a varia��o de entropia de um corpo
quando � colocado em contato com um reservat�rio t�rmico �
temperatura $T_o$. Suponha que inicialmente o corpo esteja �
temperatura $T_1$ e a capacidade t�rmica isoc�rica $C_v$ seja
constante.

\question[2.0] Obtenha a rela��o fundamental de um g�s ideal na representa��o 
de entalpia a partir da rela��o:
\[
S=N\left[s_o+c\ln \frac{U}{Nu_o}+R\ln \frac{V}{Nv_o}\right]
\]

\end{questions}


%%% Local Variables: 
%%% mode: latex
%%% TeX-master: "exame"
%%% End: 
