\documentclass[11pt]{beamer}
\date{\today}

\usetheme{Darmstadt}

\usepackage{times}
\usefonttheme{structurebold}

\usepackage[brazilian]{babel}
\usepackage{pgf,pgfarrows,pgfnodes,pgfautomata,pgfheaps}
\usepackage{amsmath,amssymb}
\usepackage[utf8]{inputenc}
\usepackage{graphicx}

\setbeamercovered{dynamic}

\title{Demonstração da Distribuição de Boltzmann}
\author{Ney Lemke}
\institute[IBB-UNESP]{%
    Departamento de Biofísica e Farmacologia}
\date{\today}                                

\colorlet{redshaded}{red!25!bg}
\colorlet{shaded}{black!25!bg}
\colorlet{shadedshaded}{black!10!bg}
\colorlet{blackshaded}{black!40!bg}

\colorlet{darkred}{red!80!black}
\colorlet{darkblue}{blue!80!black}
\colorlet{darkgreen}{green!80!black}

\begin{document}

\frame{\titlepage}

\section{Introdução}

\frame{\frametitle{Entropia de Boltzmann}
  \begin{itemize}
  \item A entropia de Boltzmann é definida como:
  \end{itemize}
  
  \begin{equation}
    S = k \log W
  \end{equation}
  
  \begin{itemize}
  \item Onde $k = 1.380662 \times 10^{-25}$ J/K (constante de Boltzmann)
  \item $W$ é a multiplicidade do sistema
  \item Na perspectiva probabilística, a entropia pode ser expressa como:
  \end{itemize}
  
  \begin{equation}
    \frac{S}{k} = -\sum_{j=1}^t p_j \ln p_j
  \end{equation}
  
  \begin{itemize}
  \item Onde $p_j$ é a probabilidade do sistema estar no estado $j$
  \end{itemize}
}

\section{Maximização da Entropia}

\frame{\frametitle{Distribuição de Boltzmann}
  \begin{itemize}
  \item Para encontrar a distribuição de Boltzmann, precisamos maximizar a entropia:
  \end{itemize}
  
  \begin{equation}
    \frac{S}{k} = -\sum_{j=1}^t p_j \ln p_j
  \end{equation}
  
  \begin{itemize}
  \item Sujeita a dois vínculos fundamentais:
  \end{itemize}
  
  \begin{equation}
    \sum_{j=1}^t p_j = 1
  \end{equation}
  
  \begin{itemize}
  \item (Normalização das probabilidades)
  \end{itemize}
  
  \begin{equation}
    \sum_{j=1}^t E_j p_j = U
  \end{equation}
  
  \begin{itemize}
  \item (Energia média do sistema)
  \end{itemize}
}

\frame{\frametitle{Método dos Multiplicadores de Lagrange}
  \begin{itemize}
  \item Utilizamos o método dos multiplicadores de Lagrange para maximizar a entropia com restrições
  \item Definimos a função de Lagrange:
  \end{itemize}
  
  \begin{equation}
    L = -\sum_{j=1}^t p_j \ln p_j - \alpha \left(\sum_{j=1}^t p_j - 1\right) - \beta \left(\sum_{j=1}^t E_j p_j - U\right)
  \end{equation}
  
  \begin{itemize}
  \item Onde $\alpha$ e $\beta$ são os multiplicadores de Lagrange
  \item Para maximizar, derivamos em relação a cada $p_i$ e igualamos a zero:
  \end{itemize}
  
  \begin{equation}
    \frac{\partial L}{\partial p_i} = -\ln p_i - 1 - \alpha - \beta E_i = 0
  \end{equation}
}

\frame{\frametitle{Resolução do Sistema}
  \begin{itemize}
  \item Da equação anterior, temos:
  \end{itemize}
  
  \begin{equation}
    -\ln p_i - 1 - \alpha - \beta E_i = 0
  \end{equation}
  
  \begin{equation}
    \ln p_i = -1 - \alpha - \beta E_i
  \end{equation}
  
  \begin{equation}
    p_i = e^{-1-\alpha-\beta E_i} = e^{-1-\alpha} \cdot e^{-\beta E_i}
  \end{equation}
  
  \begin{itemize}
  \item Definindo $e^{-1-\alpha} = \frac{1}{Z}$, onde $Z$ é a função de partição, temos:
  \end{itemize}
  
  \begin{equation}
    p_i = \frac{e^{-\beta E_i}}{Z}
  \end{equation}
}

\frame{\frametitle{Função de Partição}
  \begin{itemize}
  \item Aplicando a condição de normalização:
  \end{itemize}
  
  \begin{equation}
    \sum_{j=1}^t p_j = 1
  \end{equation}
  
  \begin{equation}
    \sum_{j=1}^t \frac{e^{-\beta E_j}}{Z} = 1
  \end{equation}
  
  \begin{equation}
    \frac{1}{Z} \sum_{j=1}^t e^{-\beta E_j} = 1
  \end{equation}
  
  \begin{itemize}
  \item Portanto, a função de partição é:
  \end{itemize}
  
  \begin{equation}
    Z = \sum_{j=1}^t e^{-\beta E_j}
  \end{equation}
}

\frame{\frametitle{Distribuição de Boltzmann}
  \begin{itemize}
  \item A distribuição de Boltzmann é, portanto:
  \end{itemize}
  
  \begin{equation}
    p_j = \frac{e^{-\beta E_j}}{Z}
  \end{equation}
  
  \begin{equation}
    Z = \sum_{j=1}^t e^{-\beta E_j}
  \end{equation}
  
  \begin{itemize}
  \item A razão entre as probabilidades de dois estados é:
  \end{itemize}
  
  \begin{equation}
    \frac{p_i}{p_j} = \frac{e^{-\beta E_i}}{e^{-\beta E_j}} = e^{-\beta(E_i-E_j)}
  \end{equation}
  
  \begin{itemize}
  \item Onde $\beta = \frac{1}{kT}$, sendo $T$ a temperatura absoluta
  \end{itemize}
}

\frame{\frametitle{Significado Físico de $\beta$}
  \begin{itemize}
  \item Para identificar $\beta$, analisamos a energia livre de Helmholtz:
  \end{itemize}
  
  \begin{equation}
    F = U - TS
  \end{equation}
  
  \begin{itemize}
  \item Podemos mostrar que:
  \end{itemize}
  
  \begin{equation}
    F = -kT\ln Z
  \end{equation}
  
  \begin{itemize}
  \item E que:
  \end{itemize}
  
  \begin{equation}
    \beta = \frac{1}{kT}
  \end{equation}
  
  \begin{itemize}
  \item Portanto, a distribuição de Boltzmann relaciona a probabilidade de ocupação de um estado com sua energia e a temperatura do sistema
  \end{itemize}
}

\frame{\frametitle{Consequências da Distribuição de Boltzmann}
  \begin{itemize}
  \item Estados de menor energia são mais prováveis
  \item A temperatura determina a dispersão das probabilidades:
    \begin{itemize}
    \item $T \rightarrow 0$: apenas o estado fundamental é ocupado
    \item $T \rightarrow \infty$: todos os estados são igualmente prováveis
    \end{itemize}
  \item Aplicações:
    \begin{itemize}
    \item Distribuição barométrica da atmosfera
    \item Distribuição de velocidades moleculares
    \item Equilíbrio químico
    \item Transições de fase
    \item Colapso de polímeros
    \end{itemize}
  \end{itemize}
}

\end{document}
