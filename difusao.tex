% $Header: /cvsroot/latex-beamer/latex-beamer/examples/beamerexample5.tex,v 1.22 2004/10/08 14:02:33 tantau Exp $

\documentclass[11pt]{beamer}

\usetheme{Darmstadt}

\usepackage{times}
\usefonttheme{structurebold}

%\usepackage[english]{babel}
\usepackage[portuges]{babel}
\usepackage{pgf,pgfarrows,pgfnodes,pgfautomata,pgfheaps}
\usepackage{amsmath,amssymb}
\usepackage[utf8]{inputenc}
\usepackage{graphicx}
\usepackage{amsfonts}

\setbeamercovered{dynamic}

\newcommand{\Lang}[1]{\operatorname{\text{\textsc{#1}}}}

\newcommand{\Class}[1]{\operatorname{\mathchoice
  {\text{\sf \small #1}}
  {\text{\sf \small #1}}
  {\text{\sf #1}}
  {\text{\sf #1}}}}

\newcommand{\NumSAT}      {\text{\small\#SAT}}
\newcommand{\NumA}        {\#_{\!A}}

\newcommand{\barA}        {\,\bar{\!A}}

\newcommand{\Nat}{\mathbb{N}}
\newcommand{\Set}[1]{\{#1\}}

\pgfdeclaremask{tu}{beamer-tu-logo-mask}
\pgfdeclaremask{computer}{beamer-computer-mask}
\pgfdeclareimage[interpolate=true,mask=computer,height=2cm]{computerimage}{beamer-computer}
\pgfdeclareimage[interpolate=true,mask=computer,height=2cm]{computerworkingimage}{beamer-computerred}
\pgfdeclareimage[mask=tu,height=.5cm]{logo}{logounesp}

\logo{\pgfuseimage{logo}}
\title{Cinética Física: Difusão, Permeação e Fluxo}
\author{Ney Lemke}
\institute[IBB-UNESP]{%
    Departamento de Biofísica e Farmacologia}
\date{\today}

\colorlet{redshaded}{red!25!bg}
\colorlet{shaded}{black!25!bg}
\colorlet{shadedshaded}{black!10!bg}
\colorlet{blackshaded}{black!40!bg}

\colorlet{darkred}{red!80!black}
\colorlet{darkblue}{blue!80!black}
\colorlet{darkgreen}{green!80!black}

\def\radius{0.96cm}
\def\innerradius{0.85cm}

\def\softness{0.4}
\definecolor{softred}{rgb}{1,\softness,\softness}
\definecolor{softgreen}{rgb}{\softness,1,\softness}
\definecolor{softblue}{rgb}{\softness,\softness,1}

\definecolor{softrg}{rgb}{1,1,\softness}
\definecolor{softrb}{rgb}{1,\softness,1}
\definecolor{softgb}{rgb}{\softness,1,1}

\AtBeginSection[]{\frame{\frametitle{Outline}\tableofcontents[current]}}

\begin{document}

\frame{\titlepage}

\section*{Outline}

\part{Parte I}

\frame{\frametitle{Outline} 
\tableofcontents[part=1]}

% Seção 1: Introdução
\section{Introdução}

\begin{frame}
    \frametitle{O que é Cinética Física?}
    \begin{itemize}
        \item Estudo das taxas de processos físicos e químicos.
        \item Foco no movimento de moléculas e partículas.
        \item Exemplos:
        \begin{itemize}
            \item Difusão de $CO_2$ em refrigerantes.
            \item Liberação controlada de fármacos.
            \item Fluxo de metabólitos através de membranas biológicas.
            \item Correntes elétricas (fluxo de íons/elétrons).
        \end{itemize}
    \end{itemize}
\end{frame}

\begin{frame}
    \frametitle{Definindo o Fluxo ($J$)}
    \begin{itemize}
        \item Quantidade de material passando por unidade de área por unidade de tempo.
        \item Consideramos fluxo em uma direção (eixo x).
        \item Material pode ser número de partículas, massa ou volume.
        \item Relação fundamental: $J = c \cdot v$, onde $c$ é a concentração e $v$ é a velocidade.
    \end{itemize}
\end{frame}

\begin{frame}
    \frametitle{Ilustração do Fluxo de Partículas}
    \begin{center}
        \includegraphics[width=0.8\textwidth]{difussion_images/image-001.png}
    \end{center}
    \begin{itemize}
        \item O fluxo de partículas é $J = cv$, onde $c$ é a concentração e $v$ é a velocidade.
        \item $A$ é a área da seção transversal.
        \item O fluido se move uma distância $\Delta x$ no tempo $\Delta t$ com velocidade $v$.
    \end{itemize}
\end{frame}

% Seção 2: Leis Lineares
\section{Leis Lineares: Forças e Fluxos}

\begin{frame}
    \frametitle{Leis Empíricas Fundamentais}
    \begin{itemize}
        \item Cinética é governada por leis empíricas.
        \item Fluxo é proporcional à força motriz.
        \item $J = Lf$, onde $L$ é a constante de proporcionalidade.
    \end{itemize}
\end{frame}

\begin{frame}
    \frametitle{Lei de Fick (Difusão)}
    \begin{itemize}
        \item Fluxo de partículas é proporcional ao gradiente de concentração.
        \item Em uma dimensão: $J = -D \frac{dc}{dx}$
        \item Em três dimensões: $\mathbf{J} = -D \nabla c$
        \item $D$: Coeficiente de difusão (maior $D$, maior fluxo para o mesmo gradiente).
        \item O sinal negativo indica fluxo no sentido de menor concentração.
    \end{itemize}
\end{frame}

\begin{frame}
    \frametitle{Outras Leis Lineares}
    \begin{itemize}
        \item \textbf{Lei de Fourier:} Fluxo de calor é proporcional ao gradiente de temperatura.
        \item \textbf{Lei de Ohm:} Fluxo de corrente elétrica é proporcional ao gradiente de potencial elétrico (tensão).
    \end{itemize}
\end{frame}

\begin{frame}
    \frametitle{A "Força" em um Gradiente de Concentração}
    \begin{itemize}
        \item Não é uma força direta em cada partícula.
        \item Resulta da tendência termodinâmica para uniformizar concentrações (entropia de mistura).
        \item Relacionada ao gradiente de potencial químico: $f = -\frac{d\mu}{dx} = -kT \frac{d \ln c}{dx} = -\frac{kT}{c}\frac{dc}{dx}$.
        \item Relação entre $D$ e mobilidade ($u$): $D = ukT$, onde $u=L/c$.
    \end{itemize}
\end{frame}

% Seção 3: A Equação da Difusão
\section{A Equação da Difusão}

\begin{frame}
    \frametitle{Derivação da Equação da Difusão}
    \begin{itemize}
        \item Combinando a Lei de Fick com a conservação de partículas.
        \item Considera o fluxo de entrada e saída de um volume elementar.
        \item Taxa de variação da concentração no tempo é igual ao negativo da divergência do fluxo: $\frac{\partial c}{\partial t} = -\frac{\partial J}{\partial x}$ (em 1D).
    \end{itemize}
\end{frame}

\begin{frame}
    \frametitle{Lei de Fick Segunda (Equação da Difusão)}
    \begin{itemize}
        \item Substituindo a Lei de Fick Primeira na relação de conservação.
        \item Em uma dimensão (D constante): $\frac{\partial c}{\partial t} = D \frac{\partial^2 c}{\partial x^2}$
        \item Em três dimensões: $\frac{\partial c}{\partial t} = D \nabla^2 c$
        \item Equação diferencial parcial que descreve $c(x,y,z,t)$.
        \item Requer condições de contorno e condição inicial para ser resolvida.
    \end{itemize}
\end{frame}

\begin{frame}
    \frametitle{Exemplo: Difusão Através de uma Lâmina/Membrana}
    \begin{itemize}
        \item Exemplo 18.1: Fluxo através de uma membrana de espessura $h$.
        \item Concentrações $c_l$ (esquerda) e $c_r$ (direita).
        \item Estado estacionário ($\partial c / \partial t = 0$).
        \item Equação se reduz a $\frac{d^2 c}{dx^2} = 0$.
        \item Perfil de concentração linear: $c(x) = A_1 x + A_2$.
        \item Condições de contorno: $c(0) = Kc_l$, $c(h) = Kc_r$ ($K$: coeficiente de partição).
        \item Perfil de concentração: $c(x) = \frac{K(c_r - c_l)}{h} x + Kc_l$.
        \item Fluxo no estado estacionário: $J = \frac{KD}{h}(c_l - c_r) = \frac{KD}{h}\Delta c$.
        \item Permeabilidade ($P$): $P = J/\Delta c = KD/h$.
    \end{itemize}
\end{frame}

\begin{frame}
    \frametitle{Exemplo: Difusão em Direção a uma Esfera}
    \begin{itemize}
        \item Partículas pequenas difundindo em direção a uma esfera de raio $a$.
        \item Taxa de colisão (corrente $I$).
        \item Resolvendo a equação da difusão em coordenadas esféricas no estado estacionário.
        \item Condições de contorno: $c(r \rightarrow \infty) = c_\infty$, $c(a) = 0$ (esfera absorvente).
        \item Perfil de concentração: $c(r) = c_\infty (1 - a/r)$.
        \item Fluxo radial: $J(r) = -D \frac{dc}{dr} = \frac{-Dc_\infty a}{r^2}$.
        \item Corrente na superfície ($r=a$): $I(a) = J(a) \cdot 4\pi a^2 = -4\pi Dc_\infty a$.
        \item Coeficiente de taxa controlada por difusão: $k_a = 4\pi Da$.
    \end{itemize}
\end{frame}

\begin{frame}
    \frametitle{Exemplo: Difusão a Partir de uma Fonte Pontual}
    \begin{itemize}
        \item Como uma gota de corante se espalha na água.
        \item Solução da equação da difusão para uma condição inicial pontual ($n_0$ na origem em $t=0$).
        \item Em uma dimensão: $c(x,t) = \frac{n_0}{(4\pi Dt)^{1/2}} \exp\left(-\frac{x^2}{4Dt}\right)$.
        \item Em $d$ dimensões: $c(r,t) = \frac{n_0}{(4\pi Dt)^{d/2}} \exp\left(-\frac{r^2}{4Dt}\right)$.
        \item A distribuição se alarga e diminui de pico com o tempo.
    \end{itemize}
\end{frame}

% Seção 4: Fontes e Sumidouros
\section{Fontes e Sumidouros}

\begin{frame}
    \frametitle{Fontes e Sumidouros na Equação da Difusão}
    \begin{itemize}
        \item Processos que criam (fontes) ou removem (sumidouros) partículas.
        \item Podem ser reações químicas, nascimentos/mortes em biologia, etc.
        \item Termos adicionados à equação da difusão: $\frac{\partial c}{\partial t} = D \nabla^2 c + \text{Fontes} - \text{Sumidouros}$.
    \end{itemize}
\end{frame}

\begin{frame}
    \frametitle{Exemplos da Biologia Populacional}
    \begin{itemize}
        \item Modelo de Malthus (crescimento exponencial): $\frac{dc}{dt} = ac$.
        \item Modelo de Verhulst (crescimento com saturação): $\frac{dc}{dt} = ac - bc^2$.
        \item Modelo de Fisher (difusão + crescimento com saturação): $\frac{\partial c}{\partial t} = D \nabla^2 c + ac - bc^2$.
        \item Mostra como a população se espalha por difusão, mas é limitada pelo crescimento.
    \end{itemize}
\end{frame}

\begin{frame}
    \frametitle{Exemplo: Difusão Acoplada a Reação Química}
    \begin{itemize}
        \item Exemplo 18.4: Fármaco difundindo de um comprimido para uma solução onde reage.
        \item Reação de depleção de primeira ordem: $\frac{dc}{dt} = -k_{rx} c$.
        \item Equação da difusão com sumidouro: $\frac{\partial c}{\partial t} = D \frac{\partial^2 c}{\partial x^2} - k_{rx} c$.
        \item No estado estacionário: $\frac{d^2 c}{dx^2} - \frac{k_{rx}}{D} c = 0$.
        \item Solução (para $c(\infty)=0$): $c(x) = c(0) \exp\left(-x\sqrt{\frac{k_{rx}}{D}}\right)$.
        \item A concentração diminui exponencialmente com a distância.
        \item Fluxo na superfície do comprimido ($x=0$): $J = c(0)\sqrt{Dk_{rx}}$.
    \end{itemize}
\end{frame}

% Seção 5: Forças Adicionais
\section{Partículas Sujeitas a Forças Adicionais}

\begin{frame}
    \frametitle{Forças Externas e Fluxos}
    \begin{itemize}
        \item Partículas podem ser sujeitas a forças direcionais (ex: gravidade, campo elétrico).
        \item Em líquidos, a velocidade é proporcional à força aplicada (regime de atrito): $f = \xi v$, onde $\xi$ é o coeficiente de atrito.
        \item Fluxo devido a uma força aplicada: $J_{ap} = c v = \frac{cf}{\xi}$.
        \item Lei de Fick generalizada (com gradiente e força): $J = -D \frac{\partial c}{\partial x} + \frac{cf}{\xi}$.
        \item Equação de Smoluchowski: $\frac{\partial c}{\partial t} = D \frac{\partial^2 c}{\partial x^2} - \frac{f}{\xi} \frac{\partial c}{\partial x}$.
    \end{itemize}
\end{frame}

% Seção 6: Equação de Einstein-Smoluchowski
\section{Equação de Einstein-Smoluchowski}

\begin{frame}
    \frametitle{Relacionando Difusão e Atrito}
    \begin{itemize}
        \item Derivada considerando o equilíbrio entre o fluxo devido à força (gravidade) e o fluxo devido ao gradiente de concentração.
        \item No equilíbrio, $J=0$.
        \item $D \frac{dc}{dx} = \frac{cf}{\xi}$.
        \item Integrando e comparando com a distribuição de Boltzmann ($c(x)/c(0) = e^{-w(x)/kT}$).
        \item \textbf{Equação de Einstein-Smoluchowski:} $D = \frac{kT}{\xi}$.
        \item Permite determinar $\xi$ a partir de $D$, dando informações sobre o tamanho/forma da partícula.
    \end{itemize}
\end{frame}

\begin{frame}
    \frametitle{Lei de Stokes e Lei de Stokes-Einstein}
    \begin{itemize}
        \item Lei de Stokes (para esferas em fluido): $\xi = 6\pi\eta a$, onde $\eta$ é a viscosidade e $a$ é o raio.
        \item \textbf{Lei de Stokes-Einstein:} $D = \frac{kT}{6\pi\eta a}$.
        \item Prevê que partículas maiores difundem mais lentamente.
        \item Coeficientes de atrito para outras formas também existem (discos, elipsoides).
    \end{itemize}
\end{frame}

\begin{frame}
    \frametitle{Perspectiva Microscópica: Movimento Browniano}
    \begin{itemize}
        \item Difusão resulta do bombardeio aleatório pelas moléculas do solvente.
        \item Modelo de caminhada aleatória (random walk).
        \item Relação entre o deslocamento quadrático médio e o coeficiente de difusão: $\langle x^2 \rangle = 2Dt$ (1D), $\langle r^2 \rangle = 2dDt$ ($d$ dimensões).
        \item Permite relacionar modelos microscópicos com medidas experimentais de $D$.
    \end{itemize}
\end{frame}

% Seção 7: Catracas Brownianas
\section{Catracas Brownianas}

\begin{frame}
    \frametitle{Convertendo Eventos de Ligação/Liberação em Movimento Direcionado}
    \begin{itemize}
        \item Como máquinas moleculares (proteínas motoras) geram movimento direcional.
        \item Combina movimento Browniano aleatório com eventos de ligação/liberação dependentes de energia, mas não direcionais.
        \item Modelo da Catraca Browniana: Partícula difundindo em um potencial assimétrico (dente de serra).
        \item Ciclo: Ligação $\rightarrow$ Liberação (difusão aleatória) $\rightarrow$ Religação (caindo no mínimo de energia mais próximo).
        \item A assimetria do potencial leva a uma probabilidade maior de se mover em uma direção.
        \item Velocidade líquida depende da assimetria e do tempo de "off" ($\tau_{off}$).
    \end{itemize}
\end{frame}

% Seção 8: Teorema Flutuação-Dissipação
\section{Teorema Flutuação-Dissipação}

\begin{frame}
    \frametitle{Relacionando Flutuações de Equilíbrio e Taxa de Abordagem ao Equilíbrio}
    \begin{itemize}
        \item Teorema importante da mecânica estatística.
        \item A magnitude das flutuações de equilíbrio está relacionada à velocidade com que o sistema atinge o equilíbrio.
        \item Permite determinar propriedades de transporte (D, $\eta$) a partir de flutuações de equilíbrio.
    \end{itemize}
\end{frame}

\begin{frame}
    \frametitle{Modelo de Langevin}
    \begin{itemize}
        \item Modelo de uma partícula em movimento Browniano em 1D.
        \item Forças atuantes: atrito ($\xi v$) e força aleatória flutuante ($f(t)$).
        \item Equação de Langevin: $m \frac{dv}{dt} = f(t) - \xi v$.
        \item $\langle f(t) \rangle = 0$.
    \end{itemize}
\end{frame}

\begin{frame}
    \frametitle{Função de Correlação Temporal da Velocidade}
    \begin{itemize}
        \item $\langle v(0)v(t) \rangle$: Média de conjunto do produto da velocidade no tempo 0 e no tempo $t$.
        \item Descreve quão rápido o movimento Browniano "apaga" a memória da velocidade inicial.
        \item Derivada do modelo de Langevin: $\langle v(0)v(t) \rangle = \frac{kT}{m} e^{-\xi t/m}$.
        \item O tempo de correlação é $m/\xi$.
    \end{itemize}
\end{frame}

\begin{frame}
    \frametitle{Relações de Green-Kubo}
    \begin{itemize}
        \item Relacionam funções de correlação de equilíbrio com coeficientes de transporte.
        \item Exemplo (coeficiente de difusão): $\int_{0}^{\infty}\langle v(0)v(t)\rangle dt = D$ (1D), $3D$ (3D).
        \item Exemplo (coeficiente de atrito): $\xi = \frac{1}{2kT}\int_{-\infty}^{\infty}\langle f(0)f(t)\rangle dt$.
        \item A viscosidade também pode ser expressa como uma função de correlação.
    \end{itemize}
\end{frame}

% Seção 9: Relações Recíprocas de Onsager
\section{Relações Recíprocas de Onsager}

\begin{frame}
    \frametitle{Descrevendo Fluxos Acoplados}
    \begin{itemize}
        \item Em sistemas com múltiplos processos de fluxo, eles podem ser acoplados.
        \item Ex: Gradiente de temperatura afeta corrente elétrica (efeito Peltier), e vice-versa (efeito Seebeck).
        \item Relações lineares generalizadas:
        \begin{align*} J_1 &= L_{11}f_1 + L_{12}f_2 \\ J_2 &= L_{21}f_1 + L_{22}f_2 \end{align*}
        \item $L_{12}$ e $L_{21}$ são coeficientes de acoplamento.
        \item Observação experimental notável: $L_{12} = L_{21}$.
    \end{itemize}
\end{frame}

\begin{frame}
    \frametitle{Princípio da Reversibilidade Microscópica}
    \begin{itemize}
        \item Explicado por L. Onsager.
        \item Em nível microscópico, os processos são reversíveis no tempo.
        \item A probabilidade média de um estado de transição em uma direção é a mesma que na direção reversa.
        \item Este princípio implica as relações recíprocas ($L_{12} = L_{21}$).
    \end{itemize}
\end{frame}

% Seção 10: Resumo
\section{Resumo}

\begin{frame}
    \frametitle{Resumo do Capítulo}
    \begin{itemize}
        \item Gradientes impulsionam fluxos (partículas, calor, carga).
        \item A taxa de fluxo é proporcional ao gradiente (Lei de Fick, Fourier, Ohm).
        \item A Equação da Difusão descreve a evolução temporal e espacial da concentração.
        \item Fontes e sumidouros (ex: reações) modificam a equação da difusão.
        \item Forças externas adicionais afetam o fluxo (Equação de Smoluchowski).
        \item A Equação de Einstein-Smoluchowski relaciona difusão e atrito ($D=kT/\xi$).
        \item A difusão microscópica é um processo de caminhada aleatória ($\langle x^2 \rangle = 2Dt$).
        \item Catracas Brownianas usam assimetria e eventos de ligação/liberação para movimento direcionado.
        \item O Teorema Flutuação-Dissipação relaciona flutuações de equilíbrio com propriedades de transporte.
        \item As Relações Recíprocas de Onsager descrevem o acoplamento entre diferentes fluxos ($L_{ij}=L_{ji}$).
    \end{itemize}
\end{frame}

% Slide Final (Opcional)
\begin{frame}
    \frametitle{Obrigado!}
    \begin{center}
        Perguntas?
    \end{center}
\end{frame}

\end{document}
