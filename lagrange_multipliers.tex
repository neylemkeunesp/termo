\documentclass{beamer}
\usepackage[utf8]{inputenc}
\usepackage{amsmath}
\usepackage{graphicx}

\title{Técnica dos Multiplicadores de Lagrange}
\author{Apresentação Gerada por IA}
\date{\today}

\begin{document}

\frame{\titlepage}

\begin{frame}
\frametitle{Introdução}
\begin{itemize}
    \item Os multiplicadores de Lagrange são uma técnica poderosa para otimização com restrições
    \item Desenvolvida por Joseph-Louis Lagrange no século XVIII
    \item Amplamente utilizada em física, engenharia e economia
\end{itemize}
\end{frame}

\begin{frame}
\frametitle{O Problema de Otimização com Restrições}
\begin{itemize}
    \item Objetivo: Maximizar ou minimizar uma função $f(x,y)$
    \item Sujeito a uma restrição $g(x,y) = c$
    \item Exemplo: Maximizar a área de um retângulo com perímetro fixo
\end{itemize}
\end{frame}

\begin{frame}
\frametitle{A Técnica dos Multiplicadores de Lagrange}
\begin{enumerate}
    \item Definir a função Lagrangiana: $L(x,y,\lambda) = f(x,y) - \lambda(g(x,y) - c)$
    \item Calcular as derivadas parciais e igualá-las a zero:
        \begin{align*}
        \frac{\partial L}{\partial x} &= 0 \\
        \frac{\partial L}{\partial y} &= 0 \\
        \frac{\partial L}{\partial \lambda} &= 0
        \end{align*}
    \item Resolver o sistema de equações
    \item Verificar os pontos críticos encontrados
\end{enumerate}
\end{frame}

\begin{frame}
\frametitle{Exemplo: Maximização da Área de um Retângulo}
\begin{itemize}
    \item Função objetivo: $f(x,y) = xy$ (área)
    \item Restrição: $g(x,y) = 2x + 2y = P$ (perímetro)
    \item Lagrangiana: $L(x,y,\lambda) = xy - \lambda(2x + 2y - P)$
    \item Solução: $x = y = P/4$ (quadrado)
\end{itemize}
\end{frame}

\begin{frame}
\frametitle{Aplicações}
\begin{itemize}
    \item Física: Princípio do trabalho virtual, mecânica analítica
    \item Engenharia: Otimização de design, controle de sistemas
    \item Economia: Maximização de utilidade sob restrições orçamentárias
    \item Aprendizado de máquina: Otimização de funções de perda com regularização
\end{itemize}
\end{frame}

\begin{frame}
\frametitle{Conclusão}
\begin{itemize}
    \item Os multiplicadores de Lagrange são uma ferramenta versátil para otimização com restrições
    \item Permitem transformar problemas com restrições em problemas sem restrições
    \item Amplamente aplicáveis em diversos campos da ciência e engenharia
    \item Compreender esta técnica é fundamental para abordar problemas complexos de otimização
\end{itemize}
\end{frame}

\end{document}
