\documentclass[fleqn,a4paper]{exam}
\addtolength{\headheight}{1\baselineskip}
\pointsinmargin
\usepackage[utf8]{inputenc}
\usepackage[portuges]{babel}
\usepackage{amsmath}
\usepackage{amssymb}
\usepackage{theorem}
\usepackage{float}
\usepackage{graphicx}
\widowpenalty=10000
\clubpenalty=10000
\raggedbottom
\newtheorem{definition}{Def.}
\theoremstyle{break} \newtheorem{exemplo}{Exemplo}
\theoremstyle{break} \newtheorem{exercicio}{Exercício}
\newcommand{\ao}{\~{a}o}
\newcommand{\cao}{\c{c}\~{a}o}
\newcommand{\coes}{\c{c}\~{o}es}
\newcommand{\cc}{\c{c}}
\newcommand{\ii}{\'{\i}}
\renewcommand{\sectionmark}[1]{\markboth{#1}{}}
\renewcommand{\subsectionmark}[1]{\markright{#1}}
\lfoot{{\small  DISTRITO DE RUBIÃO JR. S/N C.P. 510 \\
CEP  18618-000-BOTUCATU-SP }}
\rfoot{}
\cfoot{}
\lhead{{\large UNIVERSIDADE JÚLIO DE MESQUITA FILHO-UNESP} \\
INSTITUTO DE BIOCIÊNCIAS DE BOTUCATU \\
DEPARTAMENTO DE FÍSICA E BIOFÍSICA 
}
\pagestyle{empty}
\begin{document}

\begin{center}
{\large Primeira Verificação}

{\large Mecânica Estatística}
\end{center}
\vspace{0.5cm} Nome:\rule{14cm}{0.01cm} \\

\vspace{1 cm}
{\bf Só serão aceitas respostas devidamente justificadas.}
\vspace{1 cm}

\begin{questions}

% --- Questão Boltzmann clássica (adaptada de v4/v5) ---
\question[3.0] O núcleo de um átomo de hidrogênio (um próton) possui um momento magnético.
Em um campo magnético, o próton possui dois estados de diferentes energias, spin para cima e spin para baixo. Esta é a base da ressonância magnética nuclear para prótons. As populações relativas são dadas pela distribuição de Boltzmann. A diferença de energia entre os dois estados é $\Delta\epsilon =g\mu_B$ onde $g=2.79$ e $\mu=5.05\times 10^{-24}$ J Tesla$^{-1}$. Para um instrumento de ressonância nuclear, $B$=7 Tesla. Considere $T=300$ K.
\begin{parts}
  \item Calcule a função partição.
  \item Calcule a diferença populacional: $$\frac{|N_+-N_-|}{N_++N_-}$$
\end{parts}

% --- Questão de probabilidade condicional (nova, baseada em prob.tex) ---
\question[2.0] Dois dados são lançados. Qual é a probabilidade condicional de que a soma seja par, dado que a soma é maior ou igual que 8? Justifique detalhadamente.


 --- Questão combinatória (nova, baseada em prob.tex) ---
\question[2.0] Quantos subconjuntos de 3 elementos podem ser formados a partir de um conjunto com 8 elementos? E quantas listas ordenadas de 3 elementos distintos podem ser formadas? Justifique.

% --- Questão de caminhante aleatório (nova, baseada em prob.tex) ---
\question[2.0] Um caminhante aleatório dá 10 passos, cada passo pode ser para a direita (probabilidade $p=0,6$) ou para a esquerda ($q=0,4$). Qual a probabilidade de ele dar exatamente 7 passos para a direita? Use a fórmula apropriada e calcule o valor.

\question[1.0] Determine a fórmula para determinar a entropia a partir da função de partição no caso do ensemble canônico. 
\end{questions}

\end{document}
