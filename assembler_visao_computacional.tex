\documentclass[11pt]{beamer}

\usetheme{Darmstadt}

\usepackage{times}
\usefonttheme{structurebold}

\usepackage[brazilian]{babel}
\usepackage{pgf,pgfarrows,pgfnodes,pgfautomata,pgfheaps}
\usepackage{amsmath,amssymb}
\usepackage[utf8]{inputenc}
\usepackage[T1]{fontenc}
\usepackage{graphicx}
\usepackage{tikz}
\usepackage{listings}
\usepackage{xcolor}
\usepackage{hyperref}

\usetikzlibrary{arrows.meta, positioning, calc}
\setbeamercovered{dynamic}

\definecolor{codegreen}{rgb}{0,0.6,0}
\definecolor{codegray}{rgb}{0.5,0.5,0.5}
\definecolor{codepurple}{rgb}{0.58,0,0.82}
\definecolor{backcolour}{rgb}{0.95,0.95,0.92}

\lstdefinestyle{mystyle}{
    backgroundcolor=\color{backcolour},   
    commentstyle=\color{codegreen},
    keywordstyle=\color{magenta},
    numberstyle=\tiny\color{codegray},
    stringstyle=\color{codepurple},
    basicstyle=\ttfamily\footnotesize,
    breakatwhitespace=false,         
    breaklines=true,                 
    captionpos=b,                    
    keepspaces=true,                 
    numbers=left,                    
    numbersep=5pt,                  
    showspaces=false,                
    showstringspaces=false,
    showtabs=false,                  
    tabsize=2
}

\lstset{style=mystyle}

\title{Linguagem Assembly e Aplicações em Visão Computacional}
\author{Curso de Engenharia Elétrica}
\institute[UNESP]{Departamento de Engenharia Elétrica}
\date{\today}

\AtBeginSection[]{\frame{\frametitle{Conteúdo}\tableofcontents[current]}}

\begin{document}

\frame{\titlepage}

\section{Introdução à Linguagem Assembly}

\begin{frame}
\frametitle{Introdução à Linguagem Assembly}

A Linguagem Assembly (ou Linguagem de Montagem) é uma linguagem de programação de baixo nível que representa uma forma intermediária entre a linguagem de máquina (código binário) e as linguagens de alto nível (como C, Python, Java). Cada instrução em Assembly geralmente corresponde a uma única instrução de máquina, oferecendo controle direto sobre o hardware do computador.

\end{frame}

\begin{frame}
\frametitle{Características Principais}

\begin{itemize}
    \item \textbf{Proximidade ao hardware}: Permite acesso direto aos registradores do processador, memória e dispositivos de entrada/saída.
    \item \textbf{Eficiência}: Programas em Assembly podem ser extremamente eficientes quando bem otimizados.
    \item \textbf{Especificidade}: Cada família de processadores possui seu próprio conjunto de instruções Assembly.
    \item \textbf{Controle preciso}: Permite controle bit a bit e ciclo a ciclo da execução.
    \item \textbf{Complexidade}: Requer conhecimento detalhado da arquitetura do processador.
\end{itemize}

\end{frame}

\begin{frame}
\frametitle{Relevância para Engenheiros Elétricos}

Para estudantes de Engenharia Elétrica, a linguagem Assembly é particularmente relevante pelos seguintes motivos:

\begin{itemize}
    \item Desenvolvimento de sistemas embarcados e microcontroladores
    \item Programação de FPGAs e ASICs
    \item Otimização de algoritmos críticos em termos de desempenho
    \item Desenvolvimento de drivers de dispositivos
    \item Compreensão profunda da interface hardware-software
    \item Implementação de algoritmos de processamento de sinais em tempo real
\end{itemize}

\end{frame}

\section{Fundamentos da Linguagem Assembly}

\begin{frame}
\frametitle{Componentes Básicos}

\begin{itemize}
    \item \textbf{Registradores}: Pequenas unidades de memória dentro do processador
    \item \textbf{Instruções}: Comandos que o processador pode executar
    \item \textbf{Operandos}: Dados ou endereços manipulados pelas instruções
    \item \textbf{Diretivas}: Comandos para o montador (assembler), não geram código de máquina
    \item \textbf{Rótulos (Labels)}: Identificadores para posições na memória
    \item \textbf{Comentários}: Anotações no código para documentação
\end{itemize}

\end{frame}

\begin{frame}
\frametitle{Tipos de Instruções}

\begin{itemize}
    \item \textbf{Transferência de dados}: MOV, LOAD, STORE
    \item \textbf{Operações aritméticas}: ADD, SUB, MUL, DIV
    \item \textbf{Operações lógicas}: AND, OR, XOR, NOT
    \item \textbf{Controle de fluxo}: JMP, CALL, RET, CMP
    \item \textbf{Manipulação de bits}: SHL, SHR, ROL, ROR
    \item \textbf{Entrada/Saída}: IN, OUT
\end{itemize}

\end{frame}

\begin{frame}[fragile]
\frametitle{Exemplo de Código Assembly (x86)}

\begin{lstlisting}[language={[x86masm]Assembler}, basicstyle=\tiny]
section .data
    num1 dd 10      ; Primeiro numero (32 bits)
    num2 dd 20      ; Segundo numero (32 bits)
    resultado dd 0  ; Variavel para armazenar o resultado

section .text
    global _start

_start:
    mov eax, [num1]     ; Carrega o primeiro numero em EAX
    add eax, [num2]     ; Adiciona o segundo numero a EAX
    mov [resultado], eax ; Armazena o resultado
    
    ; Codigo para finalizar o programa
    mov eax, 1          ; Syscall para exit
    xor ebx, ebx        ; Codigo de retorno 0
    int 0x80            ; Interrupcao do sistema
\end{lstlisting}

\end{frame}

\section{Assembly em Sistemas Embarcados}

\begin{frame}
\frametitle{Assembly em Sistemas Embarcados}

Os engenheiros elétricos frequentemente trabalham com sistemas embarcados, onde a programação em Assembly oferece vantagens significativas:

\begin{itemize}
    \item \textbf{AVR}: Utilizado em placas Arduino
    \item \textbf{PIC}: Fabricados pela Microchip
    \item \textbf{ARM}: Amplamente utilizado em dispositivos móveis e sistemas embarcados
    \item \textbf{ESP32/ESP8266}: Para aplicações IoT
\end{itemize}

\end{frame}

\begin{frame}[fragile]
\frametitle{Exemplo de Código para ARM}

\begin{lstlisting}[language={[x86masm]Assembler}, basicstyle=\tiny]
.global _start

_start:
    LDR R0, =0x20200000  @ Endereco base do GPIO
    MOV R1, #1           @ Valor para configurar o pino como saida
    LSL R1, #18          @ Deslocar para a posicao correta
    STR R1, [R0, #4]     @ Configurar o pino 6 como saida

loop:
    MOV R1, #1           @ Valor para ligar o LED
    LSL R1, #6           @ Deslocar para o pino 6
    STR R1, [R0, #28]    @ Ligar o LED

    @ Delay
    LDR R2, =0x3F0000
delay1:
    SUBS R2, R2, #1
    BNE delay1

    MOV R1, #1           @ Valor para desligar o LED
    LSL R1, #6           @ Deslocar para o pino 6
    STR R1, [R0, #40]    @ Desligar o LED

    @ Delay
    LDR R2, =0x3F0000
delay2:
    SUBS R2, R2, #1
    BNE delay2

    B loop               @ Voltar ao inicio do loop
\end{lstlisting}

\end{frame}

\section{Visão Computacional e Assembly}

\begin{frame}
\frametitle{Por que Assembly em Visão Computacional?}

A visão computacional é uma área que demanda alto desempenho computacional, especialmente em aplicações de tempo real. O uso de Assembly neste contexto oferece:

\begin{itemize}
    \item \textbf{Processamento em tempo real}: Crucial para aplicações como sistemas de assistência ao motorista, robótica e automação industrial.
    \item \textbf{Otimização de algoritmos críticos}: Partes específicas de algoritmos de processamento de imagem podem ser otimizadas.
    \item \textbf{Acesso a instruções SIMD}: Instruções de processamento paralelo como SSE, AVX e NEON.
    \item \textbf{Controle preciso de memória}: Importante para manipulação eficiente de grandes volumes de dados de imagem.
    \item \textbf{Implementação em hardware limitado}: Sistemas embarcados com restrições de recursos.
\end{itemize}

\end{frame}

\begin{frame}
\frametitle{Instruções SIMD para Processamento de Imagens}

As instruções SIMD (Single Instruction, Multiple Data) permitem processar múltiplos pixels simultaneamente, acelerando significativamente operações de processamento de imagem:

\begin{itemize}
    \item \textbf{x86}: MMX, SSE, AVX
    \item \textbf{ARM}: NEON
    \item \textbf{MIPS}: MDMX, MSA
\end{itemize}

\end{frame}

\begin{frame}[fragile]
\frametitle{Exemplo: Conversão de RGB para Escala de Cinza com SSE}

\begin{lstlisting}[language={[x86masm]Assembler}, basicstyle=\tiny]
; Funcao para converter uma linha de pixels RGB para escala de cinza
; void rgb_to_gray_sse(unsigned char *src, unsigned char *dst, int width);
; src - ponteiro para os dados RGB (formato RGBRGBRGB...)
; dst - ponteiro para o buffer de destino em escala de cinza
; width - largura da imagem em pixels

global rgb_to_gray_sse
section .text

rgb_to_gray_sse:
    push ebp
    mov ebp, esp
    push ebx
    
    mov eax, [ebp+8]     ; src
    mov ebx, [ebp+12]    ; dst
    mov ecx, [ebp+16]    ; width
    
    ; Coeficientes para conversao RGB->Gray (formato ponto fixo)
    ; Y = 0.299*R + 0.587*G + 0.114*B
    movdqa xmm7, [r_coef] ; 0.299 em formato de ponto fixo
    movdqa xmm6, [g_coef] ; 0.587 em formato de ponto fixo
    movdqa xmm5, [b_coef] ; 0.114 em formato de ponto fixo
\end{lstlisting}

\end{frame}

\begin{frame}[fragile]
\frametitle{Exemplo: Conversão de RGB para Escala de Cinza com SSE (cont.)}

\begin{lstlisting}[language={[x86masm]Assembler}, basicstyle=\tiny]    
process_loop:
    cmp ecx, 4           ; Processar 4 pixels por vez
    jl remaining
    
    ; Carregar 4 pixels (12 bytes)
    movdqu xmm0, [eax]   ; Carregar RGBRGBRGBRGB
    
    ; Separar componentes R, G, B
    movdqa xmm1, xmm0
    movdqa xmm2, xmm0
    movdqa xmm3, xmm0
    
    ; Mascara para extrair componentes
    pand xmm1, [r_mask]  ; Extrair componente R
    pand xmm2, [g_mask]  ; Extrair componente G
    pand xmm3, [b_mask]  ; Extrair componente B
    
    ; Deslocar G e B para a posicao correta
    psrld xmm2, 8        ; Deslocar G
    psrld xmm3, 16       ; Deslocar B
\end{lstlisting}

\end{frame}

\begin{frame}[fragile]
\frametitle{Exemplo: Conversão de RGB para Escala de Cinza com SSE (cont.)}

\begin{lstlisting}[language={[x86masm]Assembler}, basicstyle=\tiny]    
    ; Multiplicar pelos coeficientes
    pmulld xmm1, xmm7    ; R * 0.299
    pmulld xmm2, xmm6    ; G * 0.587
    pmulld xmm3, xmm5    ; B * 0.114
    
    ; Somar os componentes
    paddd xmm1, xmm2
    paddd xmm1, xmm3
    
    ; Deslocar para obter o valor final (remover parte fracionaria)
    psrld xmm1, 16
    
    ; Empacotar os resultados
    packusdw xmm1, xmm1  ; Empacotar para words
    packuswb xmm1, xmm1  ; Empacotar para bytes
    
    ; Armazenar os 4 pixels resultantes
    movd [ebx], xmm1
    
    ; Atualizar ponteiros e contador
    add eax, 12          ; 3 bytes por pixel * 4 pixels
    add ebx, 4           ; 1 byte por pixel * 4 pixels
    sub ecx, 4
    jmp process_loop
\end{lstlisting}

\end{frame}

\begin{frame}[fragile]
\frametitle{Exemplo: Conversão de RGB para Escala de Cinza com SSE (cont.)}

\begin{lstlisting}[language={[x86masm]Assembler}, basicstyle=\tiny]    
remaining:
    ; Processar pixels restantes (menos de 4)
    test ecx, ecx
    jz done
    
    ; Processar um pixel por vez
single_pixel:
    movzx edx, byte [eax]    ; R
    imul edx, 299
    movzx esi, byte [eax+1]  ; G
    imul esi, 587
    add edx, esi
    movzx esi, byte [eax+2]  ; B
    imul esi, 114
    add edx, esi
    
    ; Dividir por 1000 e armazenar
    add edx, 500             ; Arredondar
    shr edx, 10              ; Dividir por 1000 (aproximadamente)
    mov [ebx], dl
    
    ; Atualizar ponteiros
    add eax, 3
    inc ebx
    dec ecx
    jnz single_pixel
    
done:
    pop ebx
    pop ebp
    ret

section .data
align 16
r_coef: dd 0x00004C8B, 0x00004C8B, 0x00004C8B, 0x00004C8B ; 0.299 * 65536
g_coef: dd 0x00009646, 0x00009646, 0x00009646, 0x00009646 ; 0.587 * 65536
b_coef: dd 0x00001D2F, 0x00001D2F, 0x00001D2F, 0x00001D2F ; 0.114 * 65536
r_mask: dd 0x000000FF, 0x000000FF, 0x000000FF, 0x000000FF
g_mask: dd 0x0000FF00, 0x0000FF00, 0x0000FF00, 0x0000FF00
b_mask: dd 0x00FF0000, 0x00FF0000, 0x00FF0000, 0x00FF0000
\end{lstlisting}

\end{frame}

\begin{frame}
\frametitle{Aplicações Práticas em Visão Computacional}

\begin{itemize}
    \item \textbf{Filtros de imagem}: Implementação otimizada de filtros como Sobel, Gaussian, etc.
    \item \textbf{Detecção de bordas}: Algoritmos como Canny podem ter partes críticas implementadas em Assembly.
    \item \textbf{Transformações geométricas}: Rotação, escala, translação de imagens.
    \item \textbf{Compressão/descompressão de imagens}: Partes do JPEG, PNG, etc.
    \item \textbf{Reconhecimento de padrões}: Aceleração de algoritmos como SIFT, SURF, ORB.
    \item \textbf{Processamento de vídeo em tempo real}: Detecção de movimento, rastreamento de objetos.
\end{itemize}

\end{frame}

\section{Sistemas de Visão Embarcados}

\begin{frame}
\frametitle{Sistemas de Visão Embarcados}

Os sistemas de visão embarcados são cada vez mais comuns em aplicações como:

\begin{itemize}
    \item \textbf{Sistemas avançados de assistência ao motorista (ADAS)}
    \item \textbf{Robôs autônomos e drones}
    \item \textbf{Sistemas de inspeção industrial}
    \item \textbf{Dispositivos médicos de imagem}
    \item \textbf{Câmeras de segurança inteligentes}
\end{itemize}

\end{frame}

\begin{frame}
\frametitle{Uso de Assembly em Sistemas Embarcados}

Nesses sistemas, o Assembly é frequentemente utilizado para:

\begin{itemize}
    \item \textbf{Rotinas de processamento críticas}: Partes do código que precisam ser extremamente eficientes
    \item \textbf{Drivers de câmera}: Controle direto do hardware de aquisição de imagem
    \item \textbf{Processamento em tempo real}: Garantir que o processamento ocorra dentro de restrições temporais rígidas
    \item \textbf{Otimização para hardware específico}: Aproveitar características únicas do processador ou DSP
\end{itemize}

\end{frame}

\begin{frame}
\frametitle{Arquitetura de Sistema de Detecção de Objetos}

\begin{center}
\begin{tikzpicture}[node distance=1.5cm, auto]
    \node[draw, rectangle, minimum width=3cm, minimum height=1cm] (camera) {Camera};
    \node[draw, rectangle, minimum width=3cm, minimum height=1cm, below of=camera] (preproc) {Pre-processamento (Assembly)};
    \node[draw, rectangle, minimum width=3cm, minimum height=1cm, below of=preproc] (feature) {Extracao de Caracteristicas};
    \node[draw, rectangle, minimum width=3cm, minimum height=1cm, below of=feature] (classify) {Classificacao};
    \node[draw, rectangle, minimum width=3cm, minimum height=1cm, below of=classify] (output) {Saida/Atuacao};
    
    \draw[->] (camera) -- (preproc);
    \draw[->] (preproc) -- (feature);
    \draw[->] (feature) -- (classify);
    \draw[->] (classify) -- (output);
    
    \node[right of=preproc, xshift=2cm] (asm) {Otimizado em Assembly};
    \node[below of=asm, yshift=1.5cm] (asm1) {- Conversao de cores};
    \node[below of=asm1, yshift=1.8cm] (asm2) {- Filtragem de ruido};
    \node[below of=asm2, yshift=1.8cm] (asm3) {- Deteccao de bordas};
    \draw[dashed] (preproc) -- (asm);
\end{tikzpicture}
\end{center}

\end{frame}

\section{Integração com Frameworks}

\begin{frame}
\frametitle{Integração com Frameworks de Alto Nível}

Embora o Assembly seja poderoso para otimizações específicas, na prática, ele é frequentemente usado em conjunto com frameworks de alto nível:

\begin{itemize}
    \item \textbf{OpenCV}: Biblioteca de visão computacional de código aberto
    \item \textbf{TensorFlow/PyTorch}: Para aplicações de aprendizado profundo
    \item \textbf{Bibliotecas específicas para sistemas embarcados}
\end{itemize}

\end{frame}

\begin{frame}
\frametitle{Abordagem Híbrida}

A abordagem mais comum é usar uma combinação de linguagens:

\begin{itemize}
    \item \textbf{C/C++}: Para a estrutura geral do programa e algoritmos complexos
    \item \textbf{Assembly}: Para otimizar funções críticas de desempenho
    \item \textbf{Python/MATLAB}: Para prototipagem e desenvolvimento inicial
\end{itemize}

\end{frame}

\begin{frame}
\frametitle{Arquitetura Híbrida}

\begin{center}
\begin{tikzpicture}[node distance=1.2cm]
    \node[draw, rectangle, minimum width=8cm, minimum height=0.8cm] (python) {Python/MATLAB (Prototipagem)};
    \node[draw, rectangle, minimum width=8cm, minimum height=0.8cm, below of=python] (cpp) {C/C++ (Implementação Principal)};
    \node[draw, rectangle, minimum width=2.5cm, minimum height=0.8cm, below of=cpp, xshift=-2.5cm] (asm1) {Assembly (Filtros)};
    \node[draw, rectangle, minimum width=2.5cm, minimum height=0.8cm, below of=cpp] (asm2) {Assembly (Conversão)};
    \node[draw, rectangle, minimum width=2.5cm, minimum height=0.8cm, below of=cpp, xshift=2.5cm] (asm3) {Assembly (Detecção)};
    \node[draw, rectangle, minimum width=8cm, minimum height=0.8cm, below of=asm2, yshift=-0.4cm] (hw) {Hardware (Processador/DSP/GPU)};
    
    \draw[->] (python) -- (cpp);
    \draw[->] (cpp) -- (asm1);
    \draw[->] (cpp) -- (asm2);
    \draw[->] (cpp) -- (asm3);
    \draw[->] (asm1) -- (hw);
    \draw[->] (asm2) -- (hw);
    \draw[->] (asm3) -- (hw);
\end{tikzpicture}
\end{center}

\end{frame}

\begin{frame}[fragile]
\frametitle{Exemplo: Função em Assembly Chamada de C++}

\begin{lstlisting}[language=C++, basicstyle=\tiny]
// Declaracao da funcao Assembly externa
extern "C" {
    void sobel_edge_asm(unsigned char *src, unsigned char *dst, int width, int height);
}

// Funcao C++ que utiliza a implementacao em Assembly
void detectEdges(cv::Mat& input, cv::Mat& output) {
    // Verificar se a imagem esta em escala de cinza
    if (input.channels() > 1) {
        cv::cvtColor(input, input, cv::COLOR_BGR2GRAY);
    }
    
    // Garantir que a imagem de saida tenha o formato correto
    output.create(input.rows, input.cols, CV_8UC1);
    
    // Chamar a funcao Assembly otimizada
    sobel_edge_asm(input.data, output.data, input.cols, input.rows);
    
    // Pos-processamento adicional, se necessario
    // cv::threshold(output, output, 50, 255, cv::THRESH_BINARY);
}
\end{lstlisting}

\end{frame}

\section{Considerações de Desempenho}

\begin{frame}
\frametitle{Gargalos de Desempenho}

Ao otimizar algoritmos de visão computacional em Assembly, é importante considerar:

\begin{itemize}
    \item \textbf{Acesso à memória}: Frequentemente o principal gargalo em processamento de imagens
    \item \textbf{Dependências de dados}: Podem limitar o paralelismo
    \item \textbf{Ramificações condicionais}: Podem causar falhas de previsão de desvio
    \item \textbf{Alinhamento de dados}: Crucial para instruções SIMD
\end{itemize}

\end{frame}

\begin{frame}
\frametitle{Técnicas de Otimização}

\begin{itemize}
    \item \textbf{Desenrolamento de loops}: Reduz a sobrecarga de controle de loop
    \item \textbf{Pré-busca de dados}: Minimiza as penalidades de cache miss
    \item \textbf{Vetorização}: Uso eficiente de instruções SIMD
    \item \textbf{Alinhamento de memória}: Garante acesso eficiente aos dados
    \item \textbf{Redução de ramificações}: Uso de instruções condicionais sem desvio
\end{itemize}

\end{frame}

\begin{frame}
\frametitle{Comparação de Desempenho}

\begin{center}
\begin{tabular}{|l|c|c|c|}
\hline
\textbf{Operação} & \textbf{C/C++} & \textbf{Assembly} & \textbf{Ganho} \\
\hline
Conversão RGB para Cinza & 100\% & 20-30\% & 3-5x \\
\hline
Filtro Gaussiano 3x3 & 100\% & 25-40\% & 2.5-4x \\
\hline
Detecção de Bordas Sobel & 100\% & 30-45\% & 2-3x \\
\hline
Limiarização & 100\% & 10-20\% & 5-10x \\
\hline
\end{tabular}
\end{center}

\small{Tempos de execução relativos (menor é melhor)}

\end{frame}

\section{Conclusão}

\begin{frame}
\frametitle{Conclusão}

A linguagem Assembly continua sendo uma ferramenta valiosa para engenheiros elétricos que trabalham com visão computacional, especialmente em sistemas embarcados e aplicações de tempo real.

\begin{itemize}
    \item Assembly oferece controle preciso sobre o hardware, permitindo otimizações que não são possíveis em linguagens de alto nível
    \item As instruções SIMD são particularmente valiosas para processamento de imagens, permitindo operações paralelas em múltiplos pixels
    \item Uma abordagem híbrida, combinando Assembly com linguagens de alto nível, geralmente oferece o melhor equilíbrio entre produtividade e desempenho
    \item O conhecimento de Assembly proporciona uma compreensão mais profunda da interface hardware-software, beneficiando engenheiros elétricos em diversas áreas
\end{itemize}

\end{frame}

\begin{frame}
\frametitle{Exercícios Propostos}

\begin{enumerate}
    \item Implemente um filtro de média 3x3 em Assembly x86 com instruções SSE.
    \item Otimize o algoritmo de detecção de bordas Sobel apresentado, utilizando instruções AVX.
    \item Desenvolva uma rotina em Assembly para controlar um sensor de imagem via interface SPI.
    \item Compare o desempenho de uma implementação de limiarização adaptativa em C++ e em Assembly.
    \item Projete um sistema embarcado simples que capture imagens de uma câmera, aplique um filtro de detecção de bordas em Assembly e exiba o resultado.
\end{enumerate}

\end{frame}

\begin{frame}
\frametitle{Referências}

\begin{thebibliography}{9}
\setbeamertemplate{bibliography item}[text]

\bibitem{intel}
Intel Corporation, 
\textit{Intel® 64 and IA-32 Architectures Software Developer's Manual}, 
2021.

\bibitem{arm}
ARM Limited, 
\textit{ARM® Architecture Reference Manual ARMv8, for ARMv8-A architecture profile}, 
2020.

\bibitem{opencv}
Bradski, G., 
\textit{The OpenCV Library}, 
Dr. Dobb's Journal of Software Tools, 
2000.

\bibitem{simd}
Lomont, C., 
\textit{Introduction to Intel Advanced Vector Extensions}, 
Intel White Paper, 
2011.

\end{thebibliography}

\end{frame}

\begin{frame}
\frametitle{Referências (cont.)}

\begin{thebibliography}{9}
\setbeamertemplate{bibliography item}[text]

\bibitem{embedded}
Yiu, J., 
\textit{The Definitive Guide to ARM® Cortex®-M3 and Cortex®-M4 Processors}, 
Newnes, 
2013.

\bibitem{vision}
Szeliski, R., 
\textit{Computer Vision: Algorithms and Applications}, 
Springer, 
2010.

\end{thebibliography}

\end{frame}

\end{document}
